%=========================================================================
% (c) Michal Bidlo, Bohuslav Křena, 2008

%TODO
%klikatelne linky
%seznam chyb predchozich kompilatoru

\definecolor{gray}{gray}{0.5}
\newcommand{\todo}[1]{\textcolor{red}{TODO\:} \textcolor{gray}{#1}}

\chapter{Introduction}\label{introduction}

Subject of this thesis project is constructing a C programming language compiler for the soft-core processor PicoBlaze-6.

    \section{Motivation}

    Currently, there is no other C compiler designed especially for the PicoBlaze-6 processor. This iteration of the design adds many useful features making the processor more suitable for programming in C therefore a compiler is a welcome addition to the existing toolchain.

    \section{The C Programming Language}

    The C programming language was created in the 1970s by Dennis Ritchie and Ken Thompson. In this project, I focused mainly on implementing its two latest standards, ISO/IEC 9899:1999 \cite{C99} and partially ISO/IEC 9899:2011 \cite{C11}.

    \section{The Previous Project}
    Similar thesis topic was elaborated by Jakub Horník as his Master thesis in 2011. This project is discussed in Section \ref{prev_pbcc}.

    While he was writing the thesis, a new version of the target processor was released (see Chapter~\ref{processor}). The resulting application is discussed in Section~\ref{prev_pbcc}.

    \section{Structure of the Document}

    After in introduction to PicoBlaze in Section \ref{processor} and explaining the basic principles of compiler design in Section \ref{theory}, the available existing and open source C language compiler front-ends are enumerated in Section \ref{compiler}.

    Next, the previous projects implementing a compiler for the older releases of the PicoBlaze processor, are listed in Section \ref{existing}.

    And finally, Section \ref{design} explains the principles that the final application builds upon and its functionality is tested and evaluated in Section \ref{evaluation}.

\chapter{The PicoBlaze Processor}\label{processor}

\emph{PicoBlaze} is a 8-bit processor created by Xilinx Inc. for their Spartan and Virtex FPGA\footnote{Field-Programmable Gate Array} series as an embeddable circuit to implement sequential programming in the parallel FPGA architecture. This means the processor is not meant to be physically manufactured, yet it would be possible \cite{PicoBlazeWeb}.

    \section{History}

    Historically, the name of the design was \texttt{KCPSM}, for Ken Chapman's Programmable State Machine and then Constant Coded Programmable State Machine. This term is now still used in the FPGA design - the \emph{VHDL}\footnote{\emph{VHSIC} (Very High Speed Integrated Circuit) Hardware Description Language} components are still caled \texttt{kcpsmX} where \texttt{X} stands for the version of the processor \cite{DSPwFPGA}.

        \subsection{PicoBlaze}

        The first version of PicoBlaze was released in May 1999.

        The initial PicoBlaze was very simple, especially compared to the current one. There was only space for 256 instructions in the program memory and there was no RAM. Only 16 8-bit registers were available to be used for storage. However, it is possible to connect an external memory through the 256 I/O pins.

        The multi-byte instructions for arithmetic (\texttt{ADDCY} and \texttt{SUBCY}) were included since the very beginning.

        There was no dedicated instruction for comparing values but both the zero and carry flags were already present \cite{PicoBlaze}.

        \subsection{PicoBlaze-2}

        The second version of the processor was initially released on 17th December 2002.

        The program could consist of 1024 instructions at most and there was no scratchpad memory for runtime variables. The instructions were able to operate only with the 32 8-bit registers and constant values.

        No changes were done to the instruction set from PicoBlaze \cite{PicoBlaze2}.

        \subsection{PicoBlaze-3}

        First release of the previous version, 3, of PicoBlaze was released on 20th May 2004. Its last revision, 7, was released in October 2003.

        The register count was reduced to 16 while their size has been kept the same. The main addition in this version is the 64 byte large scratchpad memory, that eliminates the need of an external memory connected through the I/O ports.

        Instruction set changes in this version include the addition of comparison, testing and parity counting instructions \cite{PicoBlaze3}.

        \subsection{PicoBlaze-6}

        The PicoBlaze-6 was initially released on 28th October 2010. Since then, a total of 7 newer revisions of this versions were released.

        The most recent one, PicoBlaze-6 release 8, was released on 31th March 2014 \cite{PicoBlaze6}.

        The differences and new features compared to PicoBlaze-3 are discussed in Section~\ref{kcpsm6cmp}. The main features, regardless of history, are listed in Section~\ref{mainfeatures}.

    \section{Main Features}\label{mainfeatures}

    PicoBlaze is a \emph{RISC}\footnote{Reduced Instruction Set Computing}, Harvard architecture\footnote{The program code and the data memory are stored in separate locations} processor. Every instruction is executed in 2 clock cycles \cite{PicoBlazeWeb} \cite{PicoBlaze6}.

    \begin{table}[H]
    \centering
    \begin{tabular}{ @{\extracolsep{\fill} } r | l }
        Program memory & Up to 4096 instructions \\

        Scratchpad RAM & Up to 256 bytes \\

        2 Register banks & Containing 16 8-bit registers each\\

        Input/Output & 256 8-bit ports \\

        8-bit Arithmetic-Logic Unit & Supports shifting, adding and subtracting \\
                                    & Provides AND, OR and XOR operations \\
                                    & Compares and tests \\
                                    & Implements carry and zero flags \\

        FPGA design & Can be included directly in the VHDL code \\
                    & No other equipment or code needed \\
    \end{tabular}
    \caption{PicoBlaze-6 features}
    \end{table}

    \section{Practical usage}

    A sequential processor, in comparison to parallel hardware design, is much more feasible for implementing state machines and computational cycles.

    It is suitable to used to control simple devices and to communicate over serial interfaces. The use-cases presented by the manufacturer are LCD drivers, SPI communication, controlling devices such as A/D controller.

    Time-based operations as pulse width modulation are possible with the processor, as well as display of real time clock or frequency measuring \cite{PicoBlazeExamples}.

    \section{New features and properties of PicoBlaze-6}\label{kcpsm6cmp}

    Compared to the previous version, there were major changes in PicoBlaze design after the upgrade to the sixth version, which is still backwards-compatible.

    The most important ones are covered in this section in descending order according to their impact on programming in C and the compiler itself \cite{PicoBlaze6}.

        \subsubsection{New \texttt{JUMP@} and \texttt{CALL@} Instructions}\label{jumpat}

        Indirect jumps and calls, provided the address of the function or label, are possible in the new version.

        This improves compatibility with sophisticated C programs greatly as it allows the implementation of function pointers and their calling.

        It also means there had to a new addressing mode introduced - the whole program memory cannot be covered with only an 8-bit value, so the code's section is stored in the lower four bytes of the first register and the rest of the address in the second one.

        \subsubsection{New \texttt{REGBANK} and \texttt{STAR} Instructions}

        The new version of the processor now provides two sets of 16 general purpose registers that are switchable using the \texttt{REGBANK} instruction - this means only one of the sets can be accessed at a time.

        To store values in the inactive bank, the \texttt{STAR} instruction is provided - it is used to store a value from the active bank register in another register that is in the inactive bank.

        \subsubsection{\texttt{ADDCY} and \texttt{SUBCY} Instruction Changes}

        The previous version of these instructions modified the zero flag only according to their own result. Now they add the previous zero to the current one.

        \subsubsection{New \texttt{COMPARECY} and \texttt{TESTCY} Instructions}

        Instructions added to make comparing multi-byte types easier. They store the flags (carry and zero) and propagate them according to the progress through the single bytes.

        \subsubsection{Program Memory Changes}

        Due to the two newly added memory addressing pins, it is possible to address four times more program memory (now up to 4096), increasing the possible program size and complexity.

        \subsubsection{RAM Changes}

        The amount of RAM addressable by the program was increased to up to 256 bytes from 64, yet it depends on the target device the processor will be implemented on. 

        This was achieved by modifying the opcodes of instructions of the processor, not the internals.

        \subsubsection{Call stack}

        Only 30 levels of function call depth are now available compared to 31 of PicoBlaze-3.

        \subsubsection{New \texttt{LOAD\&RETURN} Instruction and \texttt{STRING} Directive}

        The user (or compiler) can now specify a byte string location in the memory using the \texttt{STRING} directive in the assembly.

        Then, using the \texttt{LOAD\&RETURN} instruction, it is possible to load a constant value into a specified register and unconditionally return from a subroutine, making these very useful in conjunction to generate text strings to be presented to the user.

        \subsubsection{New \texttt{OUTPUTK} Instruction}

        This instruction allows the program to output a constant instead of loading it into a register and outputting it using the regular \texttt{OUTPUT}.

    \section{PacoBlaze}

    \emph{PacoBlaze} is an open-source (under the BSD license) clone of the PicoBlaze processor written in Verilog.

    Its main advantage is the possibility to be used with hardware not provided by Xilinx and modifiability and configurability. 

    Due to the higher versatility, \emph{PacoBlaze} is not as efficient (resource-wise) as the original implementation but it enables the user to remove unneeded parts from the processor and use its smallest required subset, therefore reducing FPGA space use as the result.

    The latest version is 2.2 that was released in the year 2007 so it cannot possibly implement the new instructions of PicoBlaze-6 that this project tries to use as much as possible \cite{PacoBlaze}.

    \section{PicoBlaze Assemblers}

        \subsection{pBlazASM}\label{pblazasm}

        PBlazASM is an open source assembler created by Mediatronix to be used for compiling the machine code source files directly to machine code in several formats. It can also create a representation (\texttt{.lst} files) that is then used in the pBlazSIM simulator (that is distributed in the same repository, see Section \ref{pblazsim}) \cite{PblazAsm}.

        However, the program itself is bug infested and crashes quite often. There were some trivial bugs that were possible to be fixed quickly, such as calling \texttt{free} on a pointer that was not returned by \texttt{malloc}.

    \section{PicoBlaze Simulators}

        \subsection{pBlazSIM}\label{pblazsim}

        As the debugging capabilities of the processor are limited, making debugging of the programs on real hardware too complicated, possibility of usage a simulator of the target processor is more than welcome.

        One of the most recent ones, pBlazSIM, was created by Mediatronix, too. It is actually hosted in the same repository tree as pBlazASM (see Section \ref{pblazasm}), along with some other tools \cite{PblazSim}. \cite{PblazAsmRepo}

        Hosted under the GPLv2 open source license, it is written in C++ and Qt, so its use is not limited only to one platform. Mediatronix distributes only the Windows binary form, though.

            \subsubsection{Running Under Linux}

            First attempts to run the simulator in \emph{Wine}\footnote{WINE is not an Emulator - an open source Windows API implementation} were quite unsuccessful as it was only capable to simulate the code which was included in the distributed binary package and no other.

            Attempts to compile the simulator from the source provided were also unsuccessful at first. Some changes to the distributed files were required.

            The needed steps to finish the compilation successfully are fixing the \texttt{qmake} project file to include all needed source files and adding one missing icon for the GUI\footnote{Graphical User Interface} (for example by copying the existing ones in different colors).

            The hand-compiled version is able to run the assembly files generated by pBlazASM (see Section \ref{pblazasm}) well. All features in the GUI are working, too.

            I contacted the upstream developer responsible for the changes that introduced the invalid constructions to fix them. The code was fixed by the developer in two days.

            \subsubsection{Command Line Simulator}

            A simple simulator without any user interface was introduced to the code-base recently. The officially provided build system information does not handle its compilation but due to the relative simplicity of the project, it is easy to compile this executable by hand by using the object files of the pBlaze.cpp module.

            The simplest way possible (provided you use a C++ compiler that is able to link directly) to compile this binary is by using the following command:

            \begin{center}
                \texttt{\$\{CXX\} pBlaze.cpp pBlazSIMcl.cpp -o \$\{SIM\_BINARY\}}
            \end{center}

        \subsection{PBSim}\label{pbsim}

        A project that is being developed on the University, started as a Master's project in year 2012 \cite{PbsimProj}.

        Bound to SDCC (see Section \ref{sdcc}) and tightly related to Eclipse, it is possible to use it as a part of its UI.

        It is hosted under the GPLv2 license on Github \cite{PbsimRepo} but there are no instruction to compile the project. In comparison to pBlazSIM (see Section \ref{pblazsim}), it's not active at all.

\chapter{Programming language compiler}\label{theory}

Most of the modern computer processors are programmed using a fairly complicated binary instruction set. To make writing more complex and powerful applications more pleasant and comprehensible, we are now using programming languages that are translated (\emph{compiled}) into the low level binary form executable by the processor.

In this chapter, the most vital parts and terms in construction of a compiler are discussed, with focus on C language \cite{DragonBook}.

    \section{Compiler Structure}

    A typical programming language compiler consists of two to four main parts:

    First part is an optional preprocessor which prepares the source code for the front-end, for example by removing comments or expanding macro definitions.

    A language front-end, which transforms the code from a programming language to its simple intermediate and internal representation.

    An intermediate code optimizer searches the representation for patterns that can be changed to more efficient, smaller or faster equivalents. This part is optional, too.

    And finally, the back-end, that generates the target code for the processor itself, be it assembly code, virtual machine code or a processor-native code \cite{DragonBook}.

    \begin{figure}[H]
    \centering
    \begin{tikzpicture}
        \tikzstyle{block} = [draw, rectangle, minimum width=14.5em, minimum height=2.2em]
        \node (Program) {\emph{Source Program}};
        \node[block,right=of Program,thick] (Preprocessor) {Preprocessor};
        \node[block,below=of Preprocessor,thick] (Front) {Front-End};
        \node[block,below=of Front,thick] (IMO) {Intermediate Code Optimizer};
        \node[block,below=of IMO,thick] (Back) {Back-End};
        \node[right=of Back,font=\bf,thick] (Target) {\emph{Target Code}};

        \draw[->, thick] (Program) edge (Preprocessor);
        \draw[->,thick] (Preprocessor) edge node[auto] {\emph{Stripped Source Code}} (Front.north);
        \draw[->,thick] (Front) edge node[auto] {\emph{Intermediate Code}} (IMO);
        \draw[->,thick] (IMO) edge node[auto] {\emph{Optimized Intermediate Code}} (Back);
        \draw[->,thick] (Back) edge (Target);
    \end{tikzpicture}
    \caption{Structure of a compiler front-end} \label{fig:compiler}
    \end{figure}

    \section{Preprocessor}

    Preprocessor performs a simple task of removing or replacing text in the input source code according to some pre-set rules.

    The most common task is removing comments to leave only language defined tokens for the further stages of compilation.

    In the C programming language, there are preprocessor macros, too. These serve a purpose of replacing and inserting text. However, describing the entirety of the C preprocessor is a task beyond limits of this thesis, only few directives are listed, along with their simplified descriptions \cite{GnuCppWeb}.

    \begin{table}[H]
    \begin{tabular}{r|l}
        \verb|#include "file"| or \verb|<file>| & Behaves as if the whole contents of \verb|file| were inserted \\
        & instead of it.\\
        \verb|#define MACRO ...| & Every occurrence of \verb|NAME| in the code is replaced with \\
        & what is substituted with \verb|...| (until the end of line).\\
        \verb|#if cond| & The following lines until \verb|#endif| are pasted if \verb|cond| is\\
        & met. \verb|cond| supports C expression syntax.\\
        \verb|#ifdef MACRO| & Equivalent to \verb|#if defined(MACRO)|. True if \verb|MACRO| was\\
        & defined.
    \end{tabular}
    \caption{Basic C preprocessor macros}
    \end{table}

    \section{Front-end}

        The central and most important part of the compiler, it does a whole set of operations over the source code to produce its independent representation.

        It is a language-specific part, that also checks the correctness of the input code.

        Large compiler projects even aim to its complete and clear replaceability (see Sections \ref{gcc} and \ref{llvm} for examples) to benefit from the optimization in the following stages of compilation.

        \begin{figure}[H]
        \centering
        \begin{tikzpicture}
            \tikzstyle{block} = [draw, rectangle, minimum width=10em, minimum height=2.2em]
            \node (Program) {\emph{Source Program}};
            \node[block,right=of Program,thick] (Lex) {Lexical Analyzer};
            \node[block,right=of Lex,thick] (Parser) {Parser};
            \node[block,below=of Lex,thick] (SymTab) {Symbol Table};
            \node[block,right=of SymTab,thick] (CodeGen) {Code Generator};
            \node[below=of CodeGen,font=\bf,thick] (IM) {\emph{Intermediate Code}};

            \draw[->,thick] (Program) -- (Lex);
            \draw[transform canvas={yshift=0.5ex},->,thick] (Lex) -- (Parser);
            \draw[transform canvas={yshift=-0.5ex},<-,thick] (Lex) -- (Parser);
            \draw[transform canvas={xshift=0.5ex},->,thick] (CodeGen) -- (Parser);
            \draw[transform canvas={xshift=-0.5ex},<-,thick] (CodeGen) -- (Parser);
            \draw[<->,thick] (Lex) -- (SymTab);
            \draw[<->,thick] (SymTab) -- (CodeGen);
            \draw[->,thick] (CodeGen) -- (IM);
        \end{tikzpicture}
        \caption{Structure of a compiler front-end} \label{fig:frontend}
        \end{figure}

        \subsection{Lexical analysis}

            \emph{Lexical analyzer} comes into contact with the raw source code.

            Reading the input stream character by character, it breaks it down to segments called \emph{tokens}, according to rules that are set by the language standard.

            For example, there can be a token representing a \emph{keyword} (like \texttt{if} or \texttt{for}), a literal value (\texttt{42} or \verb'"string"',...) or a type or symbol name (\texttt{foo}, \texttt{main},...) \cite{DragonBook}.

            A token of the \texttt{if} keyword will be always represented with the same string - \texttt{if}, but a variable name in C can take form of any string that matches this regular expression:\cite{C99}

            \begin{center}\texttt{[A-Za-z\_][A-Za-z\_0-9]*}\end{center}

            These names and values are then stored in the symbol table, which will be discussed in the next section.

        \subsection{Symbol Table}

            A \emph{symbol table} is a container storing names of all symbols that the lexical analyzer has detected \cite{DragonBook}.

            The values here are used further in the compilation, either to resolve the identity of tokens, or to actually assign the literal values and use them in the program itself.

        \subsection{Parser}

            \emph{Parser} then takes the stream of tokens and arranges them into a tree-like structure. This process is called \emph{syntactic analysis}. It consists not only of arranging the tree but as a side-effect, correct syntax of the input code is being checked.

            Consider the following expression for demonstration of processes taking place in the parser:

            \begin{listing}
            \centering
            \texttt{1 * foo + 5 * bar}
            \caption{Example expression}\label{lst:expr}
            \end{listing}

            The tree, as a result of the syntactic analysis, would take the form represented in the next diagram \cite{CompilerDesignInC}.

            Note the operator precedence is honored in the same way as in regular mathematical expressions.

            \begin{figure}[H]
            \centering
            \begin{tikzpicture}[node distance=1.5cm]
                \node (add) {+};
                \node (mulL) [below left of=add,xshift=-33] {*};
                \node (mulR) [below right of=add,xshift=33] {*};
                \node (1) [below left of=mulL] {\emph{value}(1)};
                \node (foo) [below right of=mulL] {\emph{identifier}(\texttt{foo})};
                \node (5) [below left of=mulR] {\emph{value}(5)};
                \node (bar) [below right of=mulR] {\emph{identifier}(\texttt{bar})};
                \coordinate (mulM) at ($(mulL)!0.5!(mulR)$);

                \draw (add) -- (mulL);
                \draw (add) -- (mulR);
                \draw (mulL) -- (1);
                \draw (mulL) -- (foo);
                \draw (mulR) -- (5);
                \draw (mulR) -- (bar);
            \end{tikzpicture}
            \caption{Syntactic tree} \label{fig:syntactic}
            \end{figure}

            There is also other information about the tokens in this expression, like types in case of variables. It is necessary to check if they are used in the right context, like if functions are used like functions and not variables or if defined operations are being used upon them. This process is called \emph{semantic analysis} \cite{DragonBook}.

            \begin{figure}[H]
            \centering
            \begin{tikzpicture}[node distance=1.5cm]
                \node (add) {+};
                \node (mulL) [below left of=add,xshift=-33] {*};
                \node (mulR) [below right of=add,xshift=33] {*};
                \node (1) [below left of=mulL] {\texttt{\textless int\textgreater} 1};
                \node (foo) [below right of=mulL] {\texttt{\textless float\textgreater foo}};
                \node (5) [below left of=mulR] {\texttt{\textless int\textgreater} 5};
                \node (bar) [below right of=mulR] {\texttt{\textless int\textgreater bar}};
                \coordinate (mulM) at ($(mulL)!0.5!(mulR)$);

                \draw (add) -- (mulL);
                \draw (add) -- (mulR);
                \draw (mulL) -- (1);
                \draw (mulL) -- (foo);
                \draw (mulR) -- (5);
                \draw (mulR) -- (bar);
            \end{tikzpicture}
            \caption{Operator and operand relation in semantic analysis} \label{fig:semantic}
            \end{figure}

            If \texttt{foo} was not a \texttt{float} but for example a \texttt{bool} variable the semantic analysis of the expression would fail because the operation of multiplication is not be defined between these two types.

            The result types of expressions, are deduced as well. In this case, in C language, the whole expression's result would take form of a \texttt{float}.

        \subsection{Code Generator}

            While the trees are being constructed and checked, the intermediate code is created as well.

            The \emph{parser} feeds the completed subtrees and their nodes into the code generator that flattens the structure into a series of instructions, similar to the assembly language of a computer processor \cite{DragonBook}.

            Only one kind of resemblance of the previous code will be left - the links in code branches for both conditional and unconditional jumps - this is required to make creating the links in the resulting code easier.

            In case the compiler does not use any kind of intermediate code, directly the target language is being emitted in this section, making the code generator serve the purpose of a back-end (described in Section \ref{desc_backend}), too.

    \section{Intermediate Code}

            As mentioned in the previous section, the intermediate code resembles assembly. The difference between the two is still quite big though. Intermediate code contains a lot more information, like variable names, information about liveness of the variables, type information and much more.

            Popular form of such output is called \emph{three address code}. Every instruction in this simple language takes a maximum of three operands, first marks where the result will be stored and the other to are the source operands \cite{DragonBook}.

            Consider the example we already used, defined in Listing \ref{lst:expr}.

            The expression will be transformed to the following (simplified) series of instructions in three address code:

            \begin{listing}\centering\begin{tabular}{l}
            \verb|mul    temp1,       1,     foo|\\
            \verb|mul    temp2,       5,     bar|\\
            \verb|add    temp1,   temp1,   temp2|
            \end{tabular}
            \caption{Example of a three address code}\label{lst:tac}
            \end{listing}

            The instructions in the code can be limited to take only two operands. The first operand will be both the result and the first argument in this case. This code is equivalent to the previous, only printed in two-operand notation:

            \begin{listing}\centering\begin{tabular}{l}
            \verb|assign  temp1,      1|\\
            \verb|mul     temp1,    foo|\\
            \verb|assign  temp2,      5|\\
            \verb|mul     temp2,    bar|\\
            \verb|add     temp1,  temp2|\\
            \end{tabular}
            \caption{Example of a two-operand intermediate code}\label{lst:2ic}
            \end{listing}

            Three address code as intermediate language is very common in compilers. This puts them close to some processor architectures, such as ARM, that uses a form of three address code as its assembly language. \cite{ARM} In contrary, other architectures, like x86 \cite{x86}, or, PicoBlaze \cite{PicoBlaze6}, that this thesis targets, use only two operands in their assembly.

        \subsection{The Intermediate Code Optimizer}

            Optimizing the code during the compilation is not absolutely vital for the whole process. However, it is a welcome part, especially by the end-users of the tool.

            The optimizer looks for patterns that can be reordered, changed for their faster equivalents or completely removed, while maintaining the same functionality as before.

            To demonstrate on the three address code from Listing \ref{lst:tac}, consider this instruction:

            \begin{center}\verb|mul    temp1,       1,     foo|\end{center}

            Multiplying anything with 1 is redundant. After removing this unnecessary operation and changing the order to preserve the same result, the example would look like this:

            \begin{listing}\centering\begin{tabular}{l}
            \verb|mul    temp1,       5,     bar|\\
            \verb|add    temp1,   temp1,     foo|\\
            \end{tabular}
            \caption{Optimized three address code from Listing \ref{lst:tac}}\label{lst:opttac}
            \end{listing}

            One whole instruction was saved in this case.

            There are many more techniques that can be applied on any amount of code. One of the simplest ones is \emph{dead code} and \emph{dead variable detection}. When a variable is not used or a conditional branch cannot be entered at any time when the program will be executed, the optimizer can completely remove it \cite{OptimizingCompiler}.

    \section{Back-end}\label{desc_backend}

        The target specific part of the compiler, generating the sought code, is the \emph{back-end}.

        For each instruction of the immediate code, it matches its equivalent, be it single instruction, or a whole function, in the target language.

        Aside from this its main task is to allocate registers for each target instruction, as the immediate code does not use any. Because the number of the registers is finite, it also needs to handle management of other data storage forms, such as the stack \cite{DragonBook}.

        More complex compilers are usually aiming to be \emph{retargetable} - that means they are equipped to support multiple back-ends for different compilation targets. There is a list of some of them in Chapter \ref{compiler}.

        Writing a compiler back-end is the goal of this thesis.

\chapter{Compiler Front-End Choice}\label{compiler}

As starting a new compiler from scratch would not be possible in the limited time frame of a bachelor thesis, I had to choose an existing front-end and provide a corresponding back-end part for the target platform.

This basically rules out all of the proprietary compilers and the selection limited to those from the world of free and open source software. In the end, the choice consisted of the following three.

    \section{GNU Compiler Collection}\label{gcc}

    The GNU Compiler Collection, more known as \texttt{GCC}, was started as a simple C compiler in 1985 by Richard Stallman. It is now one of the most widely used compiler suites not only in open source systems \cite{GccWeb}.

    Due to its huge history and background, its code base is stable and mature but it also is very hard to read due to historical reasons and the fact basically everything is wrapped in several layers of macros.

    The documentation of the inner functionality is hard to find and it is not very well arranged. Because of its heritage, the structure does not seem very transparent.

    Free Software Foundation and the GNU Project are holding governance over the development and are prohibiting major changes to the architecture or code style which drives many new developers away.\footnote{http://gcc.gnu.org/ml/gcc/2014-01/msg00176.html}

    \section{Low Level Virtual Machine}\label{llvm}

    LLVM is a modern project with a gaining popularity in past years for implementing the features very fast and providing of interesting and useful tools, like static analyzer.

    Its C and C++ front-end, Clang, is adding the latest features of the new language standards and their drafts sooner than the competitors.

    Compared to GCC, LLVM is a really young project. It was founded in 2005. The codebase is dynamically changing, written in C++ with heavy use of templates and automatically generated code \cite{LlvmWeb}.

    Its development is sponsored by companies like Google, for example to provide ability to run native applications in the browser (NaCl project, or especially its part PNaCl) or Apple, which utilizes the ecosystem in the official development toolkit provided for their products \cite{LlvmApple} \cite{LlvmNacl}.

    LLVM provides a very well documented intermediate representation of the compiled source code. Its documentation is publicly visible on their wiki page, every necessary detail is described and the community provides several easy ways to be approached.

        \subsection{Architecture}

        LLVM is strictly separated into front and back ends, divided by the LLVM intermediate code that is heavily optimized and is runnable directly in a virtual machine (hence the name Low Level \emph{Virtual Machine})

    \section{Small Device C Compiler}\label{sdcc}

    SDCC is a simple (compared to the previous two) compiler aimed to be easily retargetable and provide a quality background for creating compilers for 8bit processors \cite{SdccWeb}.

    It is not a very large project (especially when compared to GCC and LLVM) and it uses parts (for example, the preprocessor) of GCC.

    It optimizes the compiled source code with focusing on issues appearing on small devices.

    The intermediate code is not documented very well (there is a list of all the iCodes on the project's wiki) but is simple enough to be understandable.

\chapter{Existing Solutions}\label{existing}

    \section{PCComp}\label{pccomp}

    PicoBlaze C Compiler, the project of Francesco Poderico, has its own page on SourceForge\footnote{http://sourceforge.net/projects/pccomp/},
    yet there are no files to download or source code in the repository and the only relevant activity visible is a question where to actually download the compiler.

    I managed to find a Windows binary in version 1.8.4 in a web archive and a user manual describing the compiler's features, both created in 2005 or 2006.

    However, the limitations of the compiler are vast. It generates stack-based code. This is unfortunate because PicoBlaze lacks any stack \cite{pccompman}.

        \subsection{Features}

        The compiler is not strictly following the C standard and implements only its small subset.

        PicoBlaze, PicoBlaze-2 and PicoBlaze-3 support.

        Byte and word (1 and 2 bytes) types only are supported.

        One-dimensional arrays without any pointer arithmetic are implemented by the compiler.

        \subsection{Limitations}

        Type conversions are missing, as are variable modifiers (e.g. \texttt{volatile}). 

        The compiler does not support any kind of resulting code optimization, except dead branch detection.

        The compiled assembly is often buggy or even nonfunctional and the probability of getting broken code is increasing with the complexity of the input source code and the arithmetic expressions in particular.

    \section{PBCC by Bohumil Nováček}\label{not_quite_c}

    This bachelor thesis was written on Faculty of Electrical Engineering of Czech Technical University in Prague in  2008 when only PCComp (Section \ref{pccomp}) existed.

    A compiler was written as a goal of the thesis, resulting in a small application able to compile a limited subset of the C programming language \cite{PbccNovacek}.

    Also, the source code is not to be found anywhere on the Internet, only the text part of the thesis was made public.

        \subsection{Features}

        The compiler only allows the user to compile a small subset of the actual ISO/IEC 9899:1999 standard.

        Processor support is limited to PicoBlaze-3.

        The types supported are \texttt{void}, \texttt{char} and \texttt{int}.

        There is support for one-dimensional arrays only.

        Despite the simplicity of the compiler, there are some optimization methods implemented. For example, constant expressions are replaced by values directly.

        \subsection{Limitations}

        There is no support for any user-defined type, whether it is only an enumeration, a \texttt{typedef} type or a complex type (\texttt{struct} or \texttt{union}). This effectively limits the user to use only the basic types that are in this case integers sized one and two bytes.

        There is no expression conditions, strings and multidimensional arrays.

        These limitations are caused by the fact Bohumil decided to write the compiler
        from scratch without use of any framework or front-end. The time needed to finish a complete C compiler is far beyond the time-frame of a bachelor thesis.

    \section{PBCC by Jakub Horník}\label{prev_pbcc}

    PicoBlaze C Compiler is a project sponsored by Virtuální laboratoř aplikovaných mikroprocesorů
    realized on the Faculty of Information Technology, Brno University of Technology.

    It was written in years 2010 - 2011 by Jakub Horník as a part of his master's thesis and is now maintained by Zbyněk Křivka, supervisor of this thesis \cite{PbccProjekt}.

    The compiler is based on the Small Device C Compiler (SDCC) modified to provide support for the processor so it offers a subset of features of SDCC in version 3.0 \cite{PbccHornik}.

        \subsection{Features}

        There is support for adding further optimization methods provided by SDCC, additionally to its own optimization procedures 
        that are ran during the compilation process on the intermediate code.

        Data types supported are integers large from 1 to 4 bytes, there is also no problem with converting them.

        Use of arrays (even multidimensional) and pointers is implemented, including their use as function parameters.

        PicoBlaze-6 was released only a few months after the inception of the thesis that was targetting the previous one, PicoBlaze-3. This topic is discussed in Chapter \ref{processor}. The main focus of the thesis was the older iteration of the processor, therefore function pointers and all other new features were left unimplemented.

        \subsection{Limitations}

        The main reason to rewrite the compiler from scratch is to avoid carrying all the legacy instructions and features and to focus on the cleanest possible implementation of the current revision of the processor.

        Jakub also suggests allocating the registers by coloring them and using the information for better results when memory access frequency is taken in question.

        Unclear code copied over from other ports that is not very comprehensible is the reason why I wrote the whole program again while using just a few parts from the original code.

\chapter{Implementation Design}\label{design}

In this chapter, the technical details of the project are discussed.

The port itself is written in C++ with a layer wrapping the C internals of SDCC.

    \section{New Port Addition}

        As this process is not documented anywhere in the SDCC documentation and doing it properly would require deep and good understanding of the GNU autotools toolchain, the following procedure was used to add the new port to the SDCC source:

        \begin{enumerate}

        \item Create a port source directory in \texttt{src/}, in this case, I was calling it \texttt{pblaze}.

        \item Add the basic source files in the port directory, for example \texttt{main.c} and \texttt{main.h}. \texttt{main.c} has to contain an instance of \texttt{PORT} structure containing information about the port specifications and pointers to functions that will be called during the compilation.

        \item A new (unique) port ID needs to be inserted into \texttt{src/port.h}:

        \texttt{\#define TARGET\_ID\_PBLAZE    16}

        \item And create an \texttt{extern} reference to the \texttt{PORT} instance from \texttt{src/pblaze/main.c}, for example:

            \begin{verbatim}#if !OPT_DISABLE_PBLAZE
            extern PORT pblaze_port;
            #endif\end{verbatim}

        \item In \texttt{src/SDCCmain.c}, insert a reference to the structure defined in \texttt{src/pblaze/main.c}.

        \end{enumerate}

            \subsection{Automation of new port addition}

            These tasks are automated in the included \texttt{glue.sh} script. When it is executed in  Bash\footnote{Bourne Again Shell, http://www.gnu.org/software/bash/} with the \texttt{SDCC\_HOME} environment variable set to point to the directory with both PBCC and SDCC source code, it completes all the necessary tasks.

        \section{Compilation}

        After the port was added, SDCC can be compiled. The steps to achieve successful compilation are:

        \begin{enumerate}
        \item \texttt{autoconf} creates a \texttt{configure} script to configure the components and compiler options of the final binary
        \item \texttt{./configure} is a script that compiles a Makefile for the compilation itself. It is possible, for example, to modify the optimization of the compiler binary or disable compilation of ports of architectures we will not need.
        \item \texttt{make} is the compilation script itself. You can speed the whole process by using the \texttt{-jX} argument specifying that \texttt{X} compiler processes should run at the same time.
        \end{enumerate}

    \section{SDCC Internals Wrapper}

    Using pure C library calls and macros in a C++ project would be a waste of potential of the language, therefore the project is built upon a wrapper library for the SDCC internals instead of using them directly.

    The whole wrapper library is included in the \texttt{wrap} modules.

        \subsection{Approach}

            Every SDCC structure that is vital for the process is wrapped in its own class. These classes are \texttt{Set}, \texttt{EbbIndex}, \texttt{EbBlock}, \texttt{SymLink}, \texttt{Symbol}, \texttt{Value}, \texttt{Operand} and \texttt{ICode}. Their SDCC counterparts are named the same, except they have lower-case initials.

            To provide the ability of implicit up-cast, the classes are directly inheriting the structures themselves.

            Each of the classes has its methods derived from the functions and macros that are operating over them in the SDCC internal library. The methods are partially hand-written and partially generated from the definitions in the header files.

            Aside from the methods, nothing was added to the classes. None of the methods is virtual. That means the memory footprint and binary compatibility of the classes was kept.

    \section{Utilities}

    The \texttt{util} module is containing code for making the code in other parts of the projects easier to read and understand.

        \subsection{Emitter}\label{emitter}

        Every target code output in the C++ part of the port is handled using this class.

        It implements a \texttt{std::ostream}-like\footnote{Output stream class in the C++ Standard Template Library} (left shift operator overloading) API to be easy to spot in the code.

        There is one global instance of the class that is used to output from all other modules.

        The class also provides a static member variable \texttt{i} for iterating when the instruction is being output. This allows the class that is being handled to be able to check the byte position of the multi-byte operands, allowing it to change form in case it is necessary (for example in \texttt{ADD} and \texttt{ADDCY} instructions).

        \subsection{Function}

        

    \section{Register and Memory management}

    

        \subsection{Register Allocation}

        \subsection{Stack Management}

    \section{Code Generator}

    To avoid as much code duplication as possible, every instruction is being emitted using iteration over the \texttt{Emitter::i} variable.

    All instruction classes inherit from the generic \texttt{I} class that also acts as their parent and enclosing class both. It defines the \texttt{toString} virtual method that the children implement. This method is used in the overloaded left shift operator of \texttt{Emitter} (see Section \ref{emitter}) and \texttt{I} to obtain the whole string representation of the operation and write it into the output file.

        \subsection{Generating Functions and Calls}

        \subsection{Assignment Generation}

    \section{Compiled Assembly Properties}

    The compiler is producing commented assembly to be given to an assembler which then in turn produces various its binary equivalents or other formats, for example suitable for simulation.

    In this section, the vital properties, required for getting grasp of the code (for its further modification by hand, for example), are discussed.

        \subsection{Calling Conventions}

        The calling conventions were designed to utilize as much of the variables as possible because of the limited capabilities of the processor.

        Registers are saved to the function's stack by the caller, considering liveness of the variables.

        The arguments are stored consecutively in little-endian order \footnote{The least significant byte is stored first, the most significant one last} in the registers.

        For example, consider the following function:

        \begin{listing}
        \centering
        \texttt{int mul4b(int l, int r);}
        \caption{Called function prototype}\label{lst:func}
        \end{listing}

        Its initial register utilization would be as follows:

        \begin{listing}
        \centering
        \begin{tabular}{ | c | c | c | c | c | c | c | c | }
            \hline
            \texttt{s0} & \texttt{s1} & \texttt{s2} & \texttt{s3} & \texttt{s4} & \texttt{s5} & \texttt{s6} & \texttt{s7} \\
            \hline
            \texttt{l[0]} & \texttt{l[1]} & \texttt{l[2]} & \texttt{l[3]} & \texttt{r[0]} & \texttt{r[1]} & \texttt{r[2]} & \texttt{r[3]} \\
            \hline
            \hline
            \texttt{s8} & \texttt{s9} & \texttt{sA} & \texttt{sB} & \texttt{sC} & \texttt{sD} & \texttt{sE} & \texttt{sF} \\
            \hline
            free & free & free & free & free & free & free & \texttt{SP} \\
            \hline
        \end{tabular}
        \caption{Register bank state when function from Listing \ref{lst:func} is called}\label{lst:regs}
        \end{listing}

        The returned value is stored in the first registers in the same way as the parameters were stored. Little-endian order limits the register use in the caller to the bare minimum.

        Restoring of the saved registers is done in the caller, too.

        \subsection{Function Pointers}

        As mentioned in new feature overview (this particular change is discussed in Section \ref{kcpsm6cmp}), the processor now supports jumping to labels and calling functions that have their address loaded on run-time.

        To make use of as much of the new specification as possible, the compiler supports calling function pointers using the \texttt{CALL@} instruction.

        Prior to the instruction invocation itself, the function pointer is stored in the \texttt{sD} and \texttt{sE} registers. As the address of a function is larger than the size of one register, it has to be stored in two of them - \texttt{sD} contains the upper 4 bits of the address and \texttt{sE} the lower 8 bits.

        Rather than limiting all functions' arguments to reserve the two registers for a possible function pointer call, only pointers to functions with lesser than 13 bytes worth of arguments is possible.

        Other than this, there is no limitation to function pointer use.

        \subsection{Register Bank Utilization}

        From the perspective of register banks, there are two types of functions (as there are two banks). 

        The first type is either function \texttt{main} or a state when the program has not entered \texttt{main} yet. This code has access only to variables stored in registers in Bank A.

        The other type are all other functions. These have access to registers in Bank B.

        Bank selection is handled from the code of the first type. This makes the other functions able to call any other function including itself. Any kind of recursion including \texttt{main} is not supported though.

        Before and after the call, bank has to be switched to the appropriate one. Also, if the function returned any value, it has to be transferred from the other bank to the current one before the bank is switched.

        In effect, this makes having most of the program logic in the \texttt{main} function, as it does not have to store most of its registers before calling a function, thus reducing the number of necessary \texttt{FETCH} and \texttt{STORE} instructions to move the local variables in the stack memory.

        \subsection{Stack and Variable Storage}

        Local variables of the program are initially created in the registers. Only once there is not enough room for any other variable needed for an operation in future, it is freed from the register and stored on the stack.

        Stack pointer is stored only in register \texttt{sF} only in Bank B. Stack pointer is not tracked in the \texttt{main} function. As does register bank selection, this also makes recursive calls to \texttt{main} impossible.

        The \texttt{sF} register in \texttt{main} is used only as a temporary storage for loading literal values into registers in the other register bank and to access the local variables' addresses.

        \subsection{Arithmetic Operations}

        Only basic arithmetic supported by the processor is implemented, that means there is no basic support for multiplication nor division of the variables.

        The functions to multiply can be extracted from the previous PBCC (see Section \ref{prev_pbcc}) and inserted as inline assembly in a separate function, according to the calling conventions.

        Other than that, the operations are utilizing the improved instructions handling multi-byte variables in minimum cycles.

        \subsection{Comments in the Code}

        To improve the readability and comprehensiveness of the code, there are explanatory comments included in the compiled assembly.

            \subsubsection{Instruction Comments}

            Lines of most of the instructions contain a short comment explaining the instruction and its operands. This covers moves and arithmetic operations especially.

            To demonstrate, when the following code snippet is compiled:

            \begin{listing}
            \centering
            \texttt{int baz = 42 + foo - bar;}
            \caption{Example assignment}\label{lst:assign}
            \end{listing}

            Then, assuming the variable was not loaded in the registers beforehand and the other variables were already used (\texttt{foo} is in \texttt{s0} and \texttt{s1} and \texttt{bar} is in \texttt{s2} and \texttt{s3}), the resulting assembly will take the following form:

            \begin{listing}
            \centering
            \begin{verbatim}
            load    s4,     s0              ; iTemp0[0]=foo[0]
            load    s5,     s1              ; iTemp0[1]=foo[0]
            add     s4,     0x2a            ; iTemp0[0]+=(value)
                                            ; iTemp1=iTemp0
            sub     s4,     s2              ; iTemp1[0]-=bar[0]
            subcy   s5,     s3              ; iTemp1[1]-=bar[1]
                                            ; baz=iTemp1
            \end{verbatim}
            \caption{Assembly output compiled from code in Listing \ref{lst:assign}}\label{lst:assignasm}
            \end{listing}

            Each byte in the operation is covered with the regular C-style array notation.

            Also, notice only one byte of the literal value \texttt{42} (that is lower than the size of one byte) is being added to \texttt{iTemp0}. 

            Moves on temporary variables to next variables in a more complex expressions are optimized out, too. They are marked in the assembly to point out that an other variable now resides in the particular registers.

            \subsubsection{Function Comments}

            Every function is prepended with a comment that states its name with a list of its arguments. Every argument's name is written in the list along with the registers it is stored in, in case it is not stored on stack directly.

            For example, consider the following function:

            \begin{center}\texttt{char func(long arg1, int arg2, char arg3);}\end{center}

            Its representation in the assembly will be prepended with this comment:

            \begin{center}\texttt{; Function func, arguments: [arg1:\{s0,s1,s2,s3,\},arg2:\{s4,s5,\},arg3:\{s6,\},]}\end{center}

\chapter{Port Features}\label{features}

    \section{Code Clarity}

    The extensibility and readability of this port is not comparable to any other port. The code generation itself takes only about 5 lines of code for each \texttt{ICode}, compared to tens to hundreds in PBCCv2(see Section \ref{prev_pbcc}).

    Providing a C++ API also helps by providing more syntax sugar for anybody who wants to further modify the compiler. Using class methods instead of macros and functions also adds the possibility to use an IDE\footnote{Integrated Development Environment} with method suggestion.

    \section{PicoBlaze-6 Support}

    There is no other C compiler designed to produce code exclusively for the newest revision of the PicoBlaze processor.

    The most modern ones support only PicoBlaze-3 so they miss the opportunity to save not only the program space (meaning less CPU cycles for the same program) but also the scratchpad memory and registers (which means the program will utilize less CPU cycles again).

    One of the cases when this is true is multi-byte variable comparison - equality check had to be done using many jumps and storing intermediate results, the new \texttt{COMPARECY} reuses the carry and zero values that are already in the flags.

\chapter{Testing and Evaluation}\label{evaluation}

It's necessary to evaluate the overall success of a compiler project by testing it a comparing to alternative. In this section, the testing and comparison methods are described.

    \section{Testing}\label{testing}
 
    Several example source files were compiled and ran in a simulator to prove the resulting assembly corresponds to the input source code.

        \subsection{Tools and the Testing Process}

        \texttt{PBlazSIM} (described in Section \ref{pblazsim}) was chosen to simulate the resulting code due to its simplicity, maturity and liveliness of the upstream developers.

            \subsubsection{Obtaining the Simulation Result}

                The simplicity of \texttt{PBlazSIM} cleared the way for modifying of the (small) codebase and including it as a part of the test-suite in this project.

                The only modification done is as follows: When the simulation ends because of an error or reaches the correct final state of the simulation (this happens once it  the \texttt{BREAK} internal opcode is processed), it prints the state of the processor to the standard output.

                The output is formatted for easy parsing and use in the attached test suite.

            \subsubsection{Assembler}

                The input to \texttt{PBlazSIM} are rich assembly languag files enhanced with binary representation of the code it is about to simulate. This format is compiled using \texttt{PBlazASM} (see Section \ref{pblazasm}) using the \texttt{\-l} flag and results in a file with \texttt{.LST} extension.

                There were no special modifications required to get the needed results except the ones mentioned in Section \ref{pblazasm}.

        \subsection{The Framework}

        All scripts are ran recursively for each \texttt{.c} and \texttt{.tst} file in the \texttt{test} directory in the root of the code tree.

        Every \texttt{.c} file has to have its \texttt{.tst} counter-part with the same base name to be considered a valid test.

        It tests the memory and the registers for presence of any value given (with the exception of omitted zero values, considering major part of the memory will be free in most cases), be it on a specifically given position or anywhere.

        It is also possible to test \texttt{ZERO} and \texttt{CARRY} flags of the processor.

            \subsubsection{Test File Format}

            The \texttt{.tst} file format requires every such file to include declaration of every of the following variables:

            \begin{center}
            \begin{tabular}{ r | l }
                \texttt{stored\_somewhere} & \todo{...}\\

                \texttt{stored\_there} & \todo{...}\\

            \end{tabular}
            \end{center}

            The variables are declared and defined in pseudo-C style (without the type specification), that means: enumerations take a brace-enclosed list as their initializer and Boolean values can be either \texttt{true} or \texttt{false}.

            Every definition has to be ended with a semicolon. Double definitions will result in a failing test.

            For example, the following declaration will check all memory locations to contain the values 1, 2 and 3:

            \begin{center}
                \texttt{stored\_somewhere = \{1, 002, 0x3\};}
            \end{center}

            Note that the first value is in decimal system, the second one is octal and the last one hexadecimal.

        \subsection{Test Cases}

            \todo{Hodi se poznamenat, ze testy presne pod temito nazvy zatim neexistuji, jsou tak nejak random}
            \begin{center}
            \begin{tabular}{ r | l }
                \texttt{null.c} & Does not compute anything, tests if the whole stack works \\
                \texttt{basic.c} & Basic variable assignment \\
                \texttt{addsub.c} & Addition and subtraction \\
                \texttt{bitlogic.c} & Binary operation on the variables \\
                \texttt{muldiv.c} & Multiplication and division \\
                \texttt{condsimple.c} & Simple condition evaluation \\
                \texttt{condcomplex.c} & Complicated condition evaluation \\
                \texttt{function.c} & Function calling \\
                \texttt{funcpointer.c} & Function calling \\
            \end{tabular}
            \end{center}

    \section{Evaluation}

    To evaluate this project in comparison to the other alternatives (listed in Section \ref{existing}), the same test cases were used as in the functionality testing (described in Section \ref{testing}).

    The tests were designed to be compileable in all other compilers, except the cases when an unavailable feature is being tested.

    The metric of quality (in case the program compiles and is valid) was chosen to be the count of used instructions.

    A table with a matrix of test results, with a comment for the most characteristic ones follows.

    \begin{table}[H]
    \centering
    \emph{\checkmark(instructions)} for succeeded, \emph{$\times$(reason)} for failed
    \begin{tabular}{| r | c | c | c |}
    \hline
                         &        PBCCv3         &     PBCCv2 (Horník)     &       PCComp       \\
\hline
\hline\verb|basic|       &\checkmark(\textbf{50})&\checkmark               &          \checkmark\\
\hline\verb|arith|       &\checkmark(24)         &\checkmark               &$\times$(no support)\\
\hline\verb|logic|       &\checkmark(23)         &\checkmark               &   $\times$(invalid)\\
\hline\verb|cond_simple| &\checkmark             &\checkmark               &          \checkmark\\
\hline\verb|cond_complex|&\checkmark             &\checkmark               &          \checkmark\\
\hline\verb|func_basic|  &\checkmark             &\checkmark               &          \checkmark\\
\hline\verb|func_ptr|    &\checkmark             &\checkmark               &          \checkmark\\
\hline\verb|func_recurs| &\checkmark             &\checkmark               &          \checkmark\\
\hline\verb|struct|      &\checkmark             &\checkmark               &          \checkmark\\
\hline\verb|I/O|         &\checkmark             &\checkmark               &          \checkmark\\
\hline
    \end{tabular}
    \caption{Test results}
    \end{table}

\chapter{Conclusion}\label{conclusion}


%=========================================================================
